\documentclass{article}

% Language setting
% Replace `english' with e.g. `spanish' to change the document language
\usepackage[english]{babel}

% Set page size and margins
% Replace `letterpaper' with `a4paper' for UK/EU standard size
\usepackage[letterpaper,top=2cm,bottom=2cm,left=3cm,right=3cm,marginparwidth=1.75cm]{geometry}

% Useful packages
\usepackage{amsmath}
\usepackage{graphicx}
\usepackage[colorlinks=true, allcolors=blue]{hyperref}

\title{{Algorithms SV worksheet 1}}
\author{Bálint Molnár}

\begin{document}
\maketitle


\section{Complexity}

\begin{enumerate}
    \item Write down an incorrect definition for $o(n)$ by taking the definition of $O(n)$ and
replacing $\leq$ by $<$. Then find values for $k$ and $N$ that, by this definition, would
allow us to claim that $f (3n^2) \in o(n^2)$.

\item 
Prove the following equalities/inequalities:
\begin{itemize}
    \item $| \sin(n)| = O(1)$
    \item $| \sin(n)| \neq  \Theta(1)$
    \item $ 200 + \sin(n)\neq  \Theta(1)$
    \item $n^{100} = o(2^n)$
\end{itemize}
\item 
By drawing its recursion tree, Solve the following recurrence relation: $T(n)=3T(n/2)+\Theta(n).$

\end{enumerate}

\section{Sorting}

\begin{enumerate}
    \item What is the smallest number of pairwise comparisons you need to perform to
find the smallest of n items?
\item And to find the second smallest?
\item Can picking the pivot at random really make any difference to the
expected performance? How will it affect the average case? The
worst case? Discuss.
\item Consider the following method for an array of size $n$:

Split the array into groups of $5$ (for simplicity, assume that is a multiple of 5). Take the medians from all groups (i.e. the third largest from each group). Then let $x$ be the median of medians. Show that there exist constants $C_1$ and $C_2$ such that the rank of $x$ in the original array is between $C_1n$ and $C_2n$ for a large enough $n$.

\item Explain how this can be used to achieve an $O(n)$ worst-case time complexity for finding the element with rank $k$.
\item Write pseudocode for the bottom-up mergesort.
\end{enumerate}

\end{document}