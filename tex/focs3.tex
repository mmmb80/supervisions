\documentclass[11pt,a4paper]{article}
\usepackage[T1]{fontenc}
\usepackage[utf8]{inputenc}
\usepackage{hyperref}
\usepackage{enumitem}

\title{Foundations of Computer Science -- Supervision 3}
\author{}
\date{}

% Style so question body appears below label
\setlist[enumerate,1]{label=\textbf{Question \arabic*:}, leftmargin=0pt,
  itemindent=! , labelsep=1em, align=left, labelwidth=*, listparindent=0pt}
\setlist[enumerate,2]{label=(\alph*), leftmargin=2em}
\setlist[enumerate,3]{label=(\roman*), leftmargin=3em}

\begin{document}
\maketitle

\begin{enumerate}

% Q1
\item
\begin{enumerate}
    \item Code an analogue of \verb|map| and \verb|filter| for sequences.

    \item Create a lazy list of:
    \begin{enumerate}
        \item even numbers
        \item square numbers
        \item integers that are not divisible by three
    \end{enumerate}

    \item What happens if you run \verb|filter (x -> x mod 2 = 1)| on the sequence of even numbers? Why?

    \item Define a type representing rational numbers.  
    Then define a lazy list containing positive rational numbers.  
    A number may appear in the sequence multiple times, but all rational numbers should be in the sequence.

    \item (Optional) Define a lazy list containing all finite sets of integers.  
    Is it possible to define a lazy list containing all sets of integers?
\end{enumerate}

% Q2
\item
\begin{enumerate}
    \item Compare depth-first search, breadth-first search and iterative deepening.  
    What factors decide which technique should be used in a certain scenario?

    \item A lazy binary tree is either empty or is a branch containing a label and two lazy binary trees, possibly to infinite depth.  
    Present an OCaml datatype to represent lazy binary trees, along with a function that accepts a lazy binary tree and produces a lazy list that contains all of the tree’s labels.  
    The order of the elements in the lazy list does not matter, as long as it contains all potential tree nodes.

    \item Define a datatype representing lazy general trees, where nodes may have arbitrarily many (possibly infinitely many) children.

    \item How would you traverse a tree described in part (c)?

    \item (Optional) Define a lazy tree with arbitrarily long paths (i.e. for any integer $n$, there exists a path of length $n$), that has no infinite-length path.
\end{enumerate}

\end{enumerate}

\end{document}
