\documentclass{article}
\usepackage{ocgx}
% Language setting
% Replace `english' with e.g. `spanish' to change the document language
\usepackage[english]{babel}

% Set page size and margins
% Replace `letterpaper' with `a4paper' for UK/EU standard size
\usepackage[letterpaper,top=2cm,bottom=2cm,left=3cm,right=3cm,marginparwidth=1.75cm]{geometry}

% Useful packages
\usepackage{amsmath}
\usepackage{graphicx}
\usepackage[colorlinks=true, allcolors=blue]{hyperref}

\title{{Algorithms SV worksheet 3}}
\author{Bálint Molnár}

\begin{document}
\maketitle


\section{Trees}

\begin{enumerate}
   \item Prove that, in a binary search tree, if node n has two children, then
its successor has no left child.
\item Prove that, if a key is not in a bottom node, its successor, if it exists, must be.
\item Write pseudocode for insertion and deletion to a red-black tree (without converting to a B-tree)
\item What is the time complexity of merging two binary heaps?

\end{enumerate}

\section{Hash Tables}

\begin{enumerate}
    \item Make a hash table with 8 slots and insert into it the following values:
15, 23, 12, 20, 19, 8, 7, 17, 10, 11.\\
Use the hash function 
h(k) = (k mod 10) mod 8,
and resolve collisions by chaining
\item How can you handle deletions from an open addressing table? What are the
problems of the obvious naïve approach?
\item One idea to avoid handling collisions is to make the hash table large enough so that the chances of having a collision are negligible. Explain why it is not feasible.\\
\switchocg{name_int}{\textcolor{blue}{Hint:}}
\colorbox{black}{
\begin{ocg}{name_ext}{name_int}{0}
\textcolor{white}{Look up the birthday problem}
\end{ocg}}

\item Find another use of hash functions in Computer Science/Software Engineering. Explain what properties of hash functions are required for this use case, and why you also need them for hash tables.
\end{enumerate}
\section{Exam Questions}
\begin{enumerate}
    \item \url{https://www.cl.cam.ac.uk/teaching/exams/pastpapers/y2008p10q9.pdf}
    \item \url{https://www.cl.cam.ac.uk/teaching/exams/pastpapers/y2008p1q11.pdf}
\end{enumerate}


\end{document}