\documentclass{article}
\usepackage{ocgx}
% Language setting
% Replace `english' with e.g. `spanish' to change the document language
\usepackage[english]{babel}

% Set page size and margins
% Replace `letterpaper' with `a4paper' for UK/EU standard size
\usepackage[letterpaper,top=2cm,bottom=2cm,left=3cm,right=3cm,marginparwidth=1.75cm]{geometry}

% Useful packages
\usepackage{amsmath}
\usepackage{graphicx}
\usepackage[colorlinks=true, allcolors=blue]{hyperref}

\title{{Algorithms SV worksheet 4}}
\author{Bálint Molnár}

\begin{document}
\maketitle


\section{Graph Traversal}

     Let $G=(V,E)$ be a connected graph. A tree $T=(V',E')$ is a spanning tree of $G$, if
    \begin{itemize}
        \item $V=V'$
        \item $E' \subseteq E$
    \end{itemize}


     Every graph traversal generates a spanning tree: it consists of the edges that have been used to explore the graph (i.e. if node $v$ is visited from `parent' node $u$, the edge $(u,v)$ is part of the spanning tree).
\begin{enumerate}
    \item
     
    

    
   Take the following 20-node graph: Nodes are $\{ 0,\dots 19\}$, and the edges are: 
   \begin{verbatim}
       [(0, 15), (0, 14), (0, 10), (0, 17), (0, 5), (0, 19), (1, 17),
       (1, 16), (1, 5), (1, 14), (1, 4), (2, 3), (2, 9), (3, 15),
       (3, 18), (3, 9), (3, 6), (4, 12), (5, 19), (5, 6), (6, 8),
       (6, 13), (6, 11), (7, 12), (7, 11), (7, 14), (8, 17), (8, 19),
       (8, 9), (9, 11), (9, 15), (9, 12), (9, 17), (10, 16), (11, 13),
       (11, 17), (12, 17), (14, 18), (15, 17), (18, 19)]
   \end{verbatim}

    Write a DFS algorithm that starts from node $0$ and always visits its neighbours based on their numerical order (starting from the smallest).

    Calculate and visualise the algorithm's spanning tree on the graph above.
   
    \emph{Feel free to use AI tools, StackOverflow, etc. for the graph preprocessing and the visualisation, but please write the main code by yourself.}
    \item For the graph in the previous exercise, verify that for any edge $(u,v)$, that is NOT in the spanning tree, either $u$ is an ancestor of $v$, or $v$ is an ancestor of $u$ in the spanning tree. Prove that any DFS spanning tree has this property.
    \item Prove that the height of \color{red}
        some
    \color{black} BFS spanning tree is minimal (among all spanning tree heights).
    \item \emph{(Difficult:)} An edge $e$ in a connected graph $G$ is called a \emph{bridge}, if removing it disconnects the graph. Design an algorithm that finds all the bridges in linear time.\\ Hint: use the property of DFS spanning trees from Q2.
\end{enumerate}

\section{Shortest Paths}

\begin{enumerate}
    \item In a directed graph with edge weights, give a formal proof of the triangle inequality
distance($u$ to $v$) $\leq$ distance($u$ to $w$) + cost($w \rightarrow $v) for all vertices $u,v,w$ with $w \rightarrow v$.
Make sure your proof covers the cases where no path exists.
\item Prove a more general variant: distance($u$ to $v$) $\leq$ distance($u$ to $w$) + distance ($w$ to $v$).

\item Define $D_{u,v}$ for any pair of nodes, $u,v$.
Assume they satisfy the following properties:
\begin{enumerate}
    \item $D_{u,u}\leq 0$ for all $u$
    \item $D_{u,v}\leq \mathrm{cost}(u\rightarrow v)$ for all $u,v$.
    \item $D_{u,v}\leq D_{u,w} + D_{w,v}$ for all $u,v,w$
\end{enumerate}

Prove that $D_{u,v} \leq \mathrm{distance}(u\,\mathrm{to}\,v )$ for all $u,v$


\item Show that Dijkstra's algorithm eventually terminates if the graph has negative-weight edges, but no negative cycles. What is its worst-case complexity for these graphs?

\item We are given a directed graph where each edge is labelled with a weight, and where the vertices are
numbered $1,\dots,n$. Assume it contains no negative weight cycles. Define $F_{(i,j)}(k)$ to be the minimum weight path from $i$
to $j$, such that every intermediate vertex is in the set $\{1,\dots,k\}$. Give a dynamic programming equation for $F_{(i,j)}(k)$, and a
suitable definition for $F_{(i,j)}(0)$
\end{enumerate}
\end{document}