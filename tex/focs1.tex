\documentclass[11pt,a4paper]{article}
\usepackage[T1]{fontenc}
\usepackage[utf8]{inputenc}
\usepackage{hyperref}
\usepackage{enumitem}

\title{Foundations of Computer Science -- Supervision 1}
\author{}
\date{}

% Define enumerate style so body appears below label
\setlist[enumerate,1]{label=\textbf{Question \arabic*:}, leftmargin=0pt,
  itemindent=! , labelsep=1em, align=left, labelwidth=*, listparindent=0pt}

\setlist[enumerate,2]{label=(\alph*), leftmargin=2em}
\setlist[enumerate,3]{label=(\roman*), leftmargin=3em}

\begin{document}
\maketitle



\begin{enumerate}


% Q2
\item
\begin{enumerate}
    \item For each function below, state its behaviour and its time and space complexities. 
    Assume no compiler optimisation (tail-recursion doesn’t count as compiler optimisation). Justify your answers.
    \begin{enumerate}
        \item \begin{verbatim}
let rec f pred l = match l with
  | [] -> 0
  | x :: xs -> if pred x then 1 else 0 + f pred xs
\end{verbatim}

        \item \begin{verbatim}
let rec g n = if n > 1 then g (n - 1) + g (n - 2) else n
\end{verbatim}
    \end{enumerate}
\end{enumerate}

% Q3
\item 
\begin{enumerate}
    \item Write a function that takes a positive integer n and returns a list of its factors.  
    For example: \verb|factors 12 = [1;2;3;4;6;12]|.  
    The order of the factors in the list can be arbitrary, and you can assume that the argument is always positive.

    \item What is the time complexity of your algorithm? Can you think of a way to improve it?
\end{enumerate}

% Q4
\item 
\href{https://www.cl.cam.ac.uk/teaching/exams/pastpapers/y1995p1q3.pdf}{Link to past paper question}

% Q5
\item 
\begin{enumerate}
    \item Implement the following utility functions on lists.  
    Make sure that all recursive calls are tail-recursive:

    \begin{enumerate}
        \item reverse:
        \begin{verbatim}
reverse [x1;...;xn] is [xn;...;x1]
\end{verbatim}

        \item map:
        \begin{verbatim}
map f [x1;...;xn] is [f(x1);...;f(xn)]
\end{verbatim}

        \item filter: filter p l is the list containing the elements of l that 
        satisfy the predicate p, in the order that they appear in l.

        \item left fold:
        \begin{verbatim}
fold_left f a [b1;b2;...;bn]
= f (... (f (f a b1) b2) ...) bn
\end{verbatim}
    \end{enumerate}
\end{enumerate}

% Q1
\item
\begin{enumerate}
    \item Why do we need types in programming languages?

    \item What is the type of the following values? Briefly justify:
    \begin{enumerate}
        \item \begin{verbatim}
let a x y = x
\end{verbatim}
        \item \begin{verbatim}
let b x y z = if x then y else z
\end{verbatim}
        \item \begin{verbatim}
let c x y = x y
\end{verbatim}
        \item \begin{verbatim}
let rec d _ x = d [ x ] true
\end{verbatim}
    \end{enumerate}

    \item Write functions with the following types (without using type annotations):
    \begin{enumerate}
        \item \verb|'a -> 'a|
        \item \verb|int -> int list|
        \item \verb|('a -> 'b) -> ('b -> 'c) -> 'a -> 'c|
        \item \verb|('a -> 'b) -> 'a list -> 'b list|
        \item \emph{(Tricky!)} \verb|'a -> 'b| 
    \end{enumerate}
\end{enumerate}

\end{enumerate}

\end{document}
